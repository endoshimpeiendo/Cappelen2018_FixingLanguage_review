\documentclass[dvipdfmx,autodetect-engine]{jsarticle}
\usepackage{tikz}
\usepackage{biblatex}
\addbibresource{reference.bib}

\title{書評 Cappelen 2018 }
\author{某読書会
\footnote{当書読書会の参加メンバーは遠藤、岸、木下、清水、須田、高取(五十音順)}
}
\date{2019年1月25日締切 2019年2月1日公開}

\begin{document}
\maketitle

\begin{abstract}
概要を書きます。
\cite{Cappelen2018a}
\end{abstract}

\section{はじめに}
概要を示し、いくつかの批判的論点を提示する。
最後に、概念工学に関する文献レビューを付録としてつけた。

\section{本書の概要}
\paragraph{おもな主張}
いったい\emph{概念工学}とはなにか?
Cappelenによれば、\emph{表出装置(representational devices)}――を改訂あるいは評価する営みのことである。
おそらく概念工学についてのまとまった最初の書籍である本書は、概念工学の適用対象を例示したうえで、
運用上のおおまかなガイドラインである簡素な枠組みを提示する。
簡素的枠組みの特徴・美点としては、外在主義との食いあわせよし、話題の連続性よしという二点が挙げられる。

\paragraph{構成}
本書は五つのパートより構成されている。

\paragraph{I. }
[担当:]

\paragraph{II. }
[担当:]

\paragraph{III. }
[担当:]

\paragraph{IV. }
[担当:]

\paragraph{V. }
\textbf{}
[担当:]

\section{論点}
\paragraph{論点の例}
こんな感じで論点を書いたあとに[だれの論点かをここに書いておいてください 遠藤]

\paragraph{別の論点}
あるいはこんなふうに。[遠藤]

\paragraph{}

\section{総評}
ここまでの話をまとめる。
総合的な評価をくだす。


\section*{付録:関連文献紹介}
概念工学に関する文献を概観することで、当書の位置づけを理解する助けとしたい。
[岸・遠藤]

\printbibliography

\end{document}